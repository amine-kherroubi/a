% =============================================================================
% CHAPITRE 4 : ANALYSE ET CONCEPTION
% =============================================================================

\chapter{Analyse et conception}
\label{chap:conception}

\section{Introduction}

% Introduction du chapitre

\section{Démarche de conception}

% Méthodologie adoptée

\subsection{Méthode de développement}

% Méthode utilisée (Agile, V, etc.)

\subsection{Outils de modélisation}

% Outils utilisés pour la conception

\section{Architecture générale du système}

% Vue d'ensemble de l'architecture

\subsection{Architecture fonctionnelle}

% Décomposition fonctionnelle

\subsection{Architecture technique}

% Architecture technique globale

\subsection{Architecture physique}

% Déploiement et infrastructure

\section{Analyse orientée objet}

% Modélisation UML

\subsection{Diagramme de cas d'utilisation}

% Cas d'utilisation du système

% \begin{figure}[h]
%     \centering
%     \includegraphics[width=\textwidth]{use_case_diagram.png}
%     \caption{Diagramme de cas d'utilisation}
%     \label{fig:use_case}
% \end{figure}

\subsection{Diagramme de classes}

% Modèle de classes

% \begin{figure}[h]
%     \centering
%     \includegraphics[width=\textwidth]{class_diagram.png}
%     \caption{Diagramme de classes}
%     \label{fig:class_diagram}
% \end{figure}

\subsection{Diagrammes de séquence}

% Interactions entre objets

% \begin{figure}[h]
%     \centering
%     \includegraphics[width=\textwidth]{sequence_diagram.png}
%     \caption{Diagramme de séquence - Processus principal}
%     \label{fig:sequence}
% \end{figure}

\subsection{Diagramme d'activité}

% Workflow des processus

% \begin{figure}[h]
%     \centering
%     \includegraphics[width=\textwidth]{activity_diagram.png}
%     \caption{Diagramme d'activité}
%     \label{fig:activity}
% \end{figure}

\section{Conception de la base de données}

% Modélisation des données

\subsection{Modèle conceptuel de données}

% MCD

\subsection{Modèle logique de données}

% MLD

\subsection{Modèle physique de données}

% Structure des tables

\subsection{Dictionnaire des données}

% Description des entités et attributs

\section{Conception des interfaces}

% Design des interfaces utilisateur

\subsection{Charte graphique}

% Standards visuels

\subsection{Maquettes des interfaces}

% Mockups et wireframes

\subsection{Navigation et ergonomie}

% Parcours utilisateur

\section{Architecture logicielle}

% Conception technique

\subsection{Choix technologiques}

% Technologies retenues

\subsubsection{Langages de programmation}

% Justification des choix

\subsubsection{Frameworks et bibliothèques}

% Outils de développement

\subsubsection{Système de gestion de base de données}

% SGBD choisi

\subsection{Patterns architecturaux}

% Modèles d'architecture utilisés

\subsection{Organisation du code}

% Structure des modules

\section{Sécurité}

% Aspects sécuritaires

\subsection{Authentification et autorisation}

% Mécanismes de sécurité

\subsection{Protection des données}

% Chiffrement et confidentialité

\subsection{Audit et traçabilité}

% Logging et monitoring

\section{Tests et validation}

% Stratégie de test

\subsection{Plan de tests}

% Stratégie de validation

\subsection{Types de tests}

% Tests unitaires, intégration, système

\subsection{Jeux de données de test}

% Préparation des tests

\section{Conclusion}

% Synthèse de la conception