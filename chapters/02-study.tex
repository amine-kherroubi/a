% =============================================================================
% CHAPITRE 2 : ÉTUDE DE L'EXISTANT
% =============================================================================

\chapter{Étude de l'existant}
\label{chap:existant}

\section{Introduction}

% Introduction du chapitre

\section{Processus métier actuels}

% Description des processus actuels

\subsection{Workflow de génération des bilans}

% Description du processus actuel

\subsection{Acteurs impliqués}

% Identification des utilisateurs et leurs rôles

\subsection{Outils et systèmes existants}

% Présentation des outils actuellement utilisés

\section{Architecture technique actuelle}

% État de l'art technique

\subsection{Infrastructure informatique}

% Description de l'infrastructure

\subsection{Systèmes d'information en place}

% Systèmes existants

\subsection{Bases de données}

% Structure des données actuelles

\section{Analyse des performances}

% Évaluation de l'existant

\subsection{Métriques de performance}

% Indicateurs de performance actuels

\subsection{Temps de traitement}

% Analyse des délais

\subsection{Charge de travail}

% Volume de travail actuel

\section{Identification des problématiques}

% Problèmes identifiés

\subsection{Limites fonctionnelles}

% Limitations du système actuel

\subsection{Contraintes techniques}

% Problèmes techniques identifiés

\subsection{Difficultés opérationnelles}

% Problèmes rencontrés par les utilisateurs

\section{Impact sur l'activité}

% Conséquences des problèmes identifiés

\subsection{Impact sur la productivité}

% Effets sur l'efficacité

\subsection{Risques opérationnels}

% Risques identifiés

\subsection{Coûts associés}

% Coûts liés aux inefficacités

\section{Conclusion}

% Synthèse de l'analyse de l'existant