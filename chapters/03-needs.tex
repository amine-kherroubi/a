% =============================================================================
% CHAPITRE 3 : ANALYSE DES BESOINS
% =============================================================================

\chapter{Analyse des besoins}
\label{chap:besoins}

\section{Introduction}

% Introduction du chapitre

\section{Collecte des besoins}

% Méthodologie de collecte

\subsection{Entretiens avec les utilisateurs}

% Résultats des entretiens

\subsection{Observation des processus}

% Observations terrain

\subsection{Analyse documentaire}

% Étude des documents existants

\section{Expression des besoins}

% Formalisation des besoins

\subsection{Besoins fonctionnels}

% Liste des besoins fonctionnels

\subsubsection{Fonctionnalités principales}

% Fonctionnalités essentielles

\subsubsection{Fonctionnalités secondaires}

% Fonctionnalités complémentaires

\subsection{Besoins non fonctionnels}

% Exigences de qualité

\subsubsection{Performance}

% Exigences de performance

\subsubsection{Sécurité}

% Exigences de sécurité

\subsubsection{Ergonomie}

% Exigences d'utilisabilité

\subsubsection{Compatibilité}

% Exigences de compatibilité

\section{Spécifications du système cible}

% Définition du système à développer

\subsection{Vue d'ensemble du système}

% Vision globale

\subsection{Fonctionnalités détaillées}

% Description détaillée des fonctions

\subsection{Interfaces utilisateur}

% Spécifications des interfaces

\section{Contraintes et exigences}

% Contraintes identifiées

\subsection{Contraintes techniques}

% Limitations techniques

\subsection{Contraintes réglementaires}

% Conformité réglementaire

\subsection{Contraintes temporelles}

% Planning et délais

\subsection{Contraintes budgétaires}

% Limitations financières

\section{Critères d'acceptance}

% Critères de validation

\subsection{Critères fonctionnels}

% Tests d'acceptance fonctionnels

\subsection{Critères de performance}

% Tests de performance

\subsection{Critères de qualité}

% Tests de qualité

\section{Priorisation des besoins}

% Classification des besoins

\subsection{Méthode MoSCoW}

% Must have, Should have, Could have, Won't have

\subsection{Matrice de priorisation}

% Importance vs complexité

\section{Conclusion}

% Synthèse des besoins identifiés