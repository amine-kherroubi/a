% =============================================================================
% ANNEXES
% =============================================================================

\begin{appendices}

    % =============================================================================
    % ANNEXE A : CAPTURES D'ÉCRAN DÉTAILLÉES
    % =============================================================================
    \chapter{Captures d'écran détaillées}
    \label{app:screenshots}

    % Ajoutez ici des captures d'écran supplémentaires de l'application

    \section{Interface principale}

    % \begin{figure}[h]
    %     \centering
    %     \includegraphics[width=\textwidth]{screenshot_main.png}
    %     \caption{Interface principale de l'application}
    %     \label{fig:screenshot_main}
    % \end{figure}

    \section{Interfaces de configuration}

    % \begin{figure}[h]
    %     \centering
    %     \includegraphics[width=\textwidth]{screenshot_config.png}
    %     \caption{Interface de configuration}
    %     \label{fig:screenshot_config}
    % \end{figure}

    \section{Rapports générés}

    % \begin{figure}[h]
    %     \centering
    %     \includegraphics[width=\textwidth]{screenshot_report.png}
    %     \caption{Exemple de rapport généré}
    %     \label{fig:screenshot_report}
    % \end{figure}

    % =============================================================================
    % ANNEXE B : EXTRAITS DE CODE SIGNIFICATIFS
    % =============================================================================
    \chapter{Extraits de code significatifs}
    \label{app:code}

    \section{Algorithmes principaux}

    % Ajoutez ici des extraits de code importants

    \begin{lstlisting}[language=Java, caption=Exemple d'algorithme principal]
// Insérez ici votre code exemple
public class ExempleAlgorithme {
    // Code d'exemple
}
\end{lstlisting}

    \section{Structure des données}

    % Structure des classes principales

    \begin{lstlisting}[language=SQL, caption=Structure de base de données]
-- Exemple de structure de table
CREATE TABLE exemple_table (
    id INTEGER PRIMARY KEY,
    nom VARCHAR(255) NOT NULL,
    date_creation DATETIME DEFAULT CURRENT_TIMESTAMP
);
\end{lstlisting}

    % =============================================================================
    % ANNEXE C : DOCUMENTATION TECHNIQUE
    % =============================================================================
    \chapter{Documentation technique complémentaire}
    \label{app:technical_doc}

    \section{Configuration système}

    % Prérequis et configuration

    \subsection{Prérequis matériels}

    % Configuration minimale requise

    \subsection{Prérequis logiciels}

    % Dépendances logicielles

    \section{Guide d'installation}

    % Procédure d'installation détaillée

    \subsection{Installation de l'environnement}

    % Étapes d'installation

    \subsection{Configuration initiale}

    % Paramétrage initial

    \section{Guide de déploiement}

    % Instructions de mise en production

    \subsection{Préparation de l'environnement de production}

    % Configuration serveur

    \subsection{Procédure de déploiement}

    % Étapes de déploiement

    % =============================================================================
    % ANNEXE D : RÉSULTATS DE TESTS
    % =============================================================================
    \chapter{Résultats de tests détaillés}
    \label{app:test_results}

    \section{Tests unitaires}

    % Résultats des tests unitaires

    \section{Tests d'intégration}

    % Résultats des tests d'intégration

    \section{Tests de performance}

    % Métriques de performance

    % Exemple de tableau de résultats
    \begin{table}[h]
        \centering
        \begin{tabular}{|l|c|c|c|}
            \hline
            \textbf{Test} & \textbf{Temps (ms)} & \textbf{Statut} & \textbf{Commentaires} \\
            \hline
            Test 1        & 150                 & Réussi          & -                     \\
            Test 2        & 320                 & Réussi          & -                     \\
            Test 3        & 89                  & Réussi          & -                     \\
            \hline
        \end{tabular}
        \caption{Exemple de résultats de tests}
        \label{tab:test_results}
    \end{table}

    % =============================================================================
    % ANNEXE E : QUESTIONNAIRES ET ENTRETIENS
    % =============================================================================
    \chapter{Questionnaires et entretiens utilisateurs}
    \label{app:surveys}

    \section{Questionnaire de satisfaction}

    % Questionnaire utilisé pour évaluer la satisfaction

    \section{Grille d'entretien}

    % Questions posées lors des entretiens

    \section{Synthèse des réponses}

    % Analyse des retours utilisateurs

\end{appendices}