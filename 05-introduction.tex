\chapter*{Introduction générale}
\addcontentsline{toc}{chapter}{Introduction générale}
\markboth{Introduction générale}{Introduction générale}
\label{chap:introduction}

\section*{Contexte}

La digitalisation des processus bancaires représente aujourd'hui un enjeu majeur pour les institutions financières algériennes. Dans ce contexte de transformation numérique, la \textbf{Banque Nationale de l'Habitat (BNH)}, anciennement Caisse Nationale du Logement (CNL), s'engage dans une démarche d'optimisation de ses systèmes d'information pour améliorer l'efficacité de ses services, notamment dans la gestion des aides au logement rural.

\medskip

La BNH, en tant qu'établissement public à caractère économique et commercial, joue un rôle central dans le financement du logement en Algérie. L'institution gère notamment les programmes d'aides au logement rural, dispositifs essentiels de la politique nationale d'habitat qui nécessitent un suivi rigoureux et une reporting précis pour évaluer leur impact et leur efficacité sur le terrain.

\medskip

La Direction des Systèmes d'Information de la BNH fait face à des défis croissants liés au volume important de données à traiter et à la nécessité de produire des bilans mensuels détaillés concernant l'attribution et le suivi des aides au logement rural. Ces bilans constituent des outils de pilotage indispensables pour les décideurs et les organismes de tutelle.

\section*{Problématique}

Le processus actuel de génération des bilans mensuels relatifs aux aides au logement rural présente plusieurs limitations importantes qui impactent l'efficacité opérationnelle de la BNH.

\medskip

Premièrement, la production de ces bilans repose sur des procédures largement manuelles, impliquant la collecte de données depuis diverses sources, leur traitement dans des fichiers Excel séparés, et leur consolidation manuelle. Cette approche génère des risques d'erreurs significatifs, des incohérences dans les données, et consume un temps considérable des équipes métier.

\medskip

Deuxièmement, les délais de production des bilans mensuels sont devenus incompatibles avec les exigences de réactivité imposées par l'environnement bancaire moderne. Le processus manuel actuel peut prendre plusieurs semaines, retardant ainsi la prise de décisions stratégiques et la communication d'informations aux parties prenantes.

\medskip

Troisièmement, l'absence d'automatisation limite la capacité de la banque à effectuer des analyses approfondies et à identifier rapidement les tendances dans l'attribution des aides au logement rural. Cette situation entrave la mise en œuvre d'une politique d'amélioration continue des services offerts aux bénéficiaires.

\medskip

Face à ces constats, la question centrale qui se pose est : comment développer un système automatisé permettant de générer efficacement et de manière fiable les bilans mensuels relatifs aux aides au logement rural, tout en respectant les contraintes réglementaires et les spécifications métier de la BNH ?

\section*{Objectifs}

Ce projet de fin d'études vise à concevoir et développer un outil informatique d'automatisation de la génération des bilans mensuels relatifs aux aides au logement rural pour la Banque Nationale de l'Habitat.

\medskip

Les objectifs spécifiques de ce travail sont les suivants :

\medskip

\textbf{Objectifs techniques :}
\begin{itemize}
    \item Analyser et modéliser les processus métier actuels de génération des bilans mensuels
    \item Concevoir une architecture technique robuste et évolutive pour l'outil d'automatisation
    \item Développer une application web responsive permettant la génération automatisée des bilans
    \item Intégrer l'outil avec l'écosystème technologique existant de la BNH
    \item Mettre en place des mécanismes de validation et de contrôle de qualité des données
\end{itemize}

\textbf{Objectifs fonctionnels :}
\begin{itemize}
    \item Automatiser la collecte et l'agrégation des données relatives aux aides au logement rural
    \item Générer automatiquement les différents types de bilans mensuels requis
    \item Fournir des interfaces utilisateur intuitives pour les différents profils d'utilisateurs
    \item Permettre l'export des bilans dans différents formats (PDF, Excel, etc.)
    \item Implémenter des fonctionnalités de planification et d'historisation des bilans
\end{itemize}

\textbf{Objectifs organisationnels :}
\begin{itemize}
    \item Réduire significativement les délais de production des bilans mensuels
    \item Minimiser les risques d'erreurs liés aux processus manuels
    \item Améliorer la qualité et la cohérence des rapports produits
    \item Libérer du temps métier pour des tâches à plus forte valeur ajoutée
    \item Faciliter la prise de décision grâce à une information plus rapide et fiable
\end{itemize}

\section*{Organisation du mémoire}

Ce mémoire est structuré en cinq chapitres principaux qui retracent l'ensemble de la démarche adoptée pour mener à bien ce projet.

\medskip

Le premier chapitre \og \textbf{Présentation de la Banque Nationale de l'Habitat} \fg présente l'organisme d'accueil, ses missions, son organisation, et plus particulièrement la Direction des Systèmes d'Information dans laquelle s'inscrit ce projet. Il expose également le contexte des aides au logement rural et les enjeux de leur digitalisation.

\medskip

Le deuxième chapitre \og \textbf{Étude de l'existant et analyse des besoins} \fg analyse en détail les processus actuels de génération des bilans mensuels, identifie les problématiques et limites du système existant, et définit les besoins fonctionnels et techniques du système cible.

\medskip

Le troisième chapitre \og \textbf{Analyse et conception de la solution} \fg présente la démarche de conception adoptée, l'architecture technique retenue, les choix technologiques justifiés, ainsi que la modélisation UML du système à développer.

\medskip

Le quatrième chapitre \og \textbf{Développement et réalisation} \fg détaille la phase de développement, présente la méthodologie de travail adoptée, les technologies utilisées, et expose les principales fonctionnalités réalisées de l'application.

\medskip

Le cinquième chapitre \og \textbf{Résultats et évaluation} \fg présente les résultats obtenus, les tests effectués, les gains mesurés en termes d'efficacité et de qualité, ainsi que les retours des utilisateurs finaux.

\medskip

Enfin, une conclusion générale synthétise les réalisations du projet, présente un bilan critique du travail accompli, et propose des perspectives d'évolution pour l'outil développé et les processus métier de la BNH.