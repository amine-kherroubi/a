\chapter{Présentation de la Banque Nationale de l'Habitat}
\clearpage
\label{chap:presentation}

\section{Introduction}

Ce chapitre présente le cadre institutionnel et organisationnel de notre stage de fin d'études effectué au sein de la Banque Nationale de l'Habitat (BNH). Nous y exposons l'historique et les missions de cette institution bancaire spécialisée, son organisation structurelle, ainsi que l'environnement technologique de la Direction des Systèmes d'Information où s'est déroulé notre projet. Une attention particulière est accordée au contexte des aides au logement rural et aux enjeux de modernisation des processus de gestion associés.

\section{Historique et missions de la BNH}

\subsection{Évolution institutionnelle}

La Banque Nationale de l'Habitat (BNH) trouve ses origines dans la Caisse Nationale du Logement (CNL), créée par décret n° 91-145 du 12 mai 1991. Cette institution publique avait pour mission initiale de gérer et financer les programmes nationaux de logement social et d'habitat rural.

\medskip

En 2018, dans le cadre de la modernisation du secteur bancaire algérien et de l'optimisation de la gouvernance des établissements publics financiers, la CNL a été transformée en Banque Nationale de l'Habitat par décret exécutif n° 18-96 du 2 avril 2018. Cette transformation marque une évolution stratégique vers un modèle bancaire intégré, capable de proposer une gamme élargie de services financiers tout en conservant sa vocation sociale originelle.

\subsection{Missions principales}

La BNH, en tant qu'établissement public à caractère économique et commercial, assume plusieurs missions fondamentales dans l'écosystème financier algérien :

\medskip

\textbf{Financement de l'habitat :} La banque constitue l'acteur principal du financement du logement en Algérie, gérant les dispositifs de soutien à l'accession à la propriété, notamment le programme d'aide à l'accession à la propriété immobilière (AAPL) et les formules de location-vente.

\textbf{Gestion des aides au logement rural :} L'institution administre les programmes gouvernementaux d'aide à l'habitat rural, visant à améliorer les conditions de logement des populations rurales et à réduire les disparités territoriales en matière d'habitat.

\textbf{Services bancaires spécialisés :} Au-delà de sa mission sociale, la BNH développe une offre de services bancaires traditionnels adaptés aux besoins de sa clientèle, incluant les comptes courants, l'épargne, et les crédits à la consommation.

\textbf{Développement territorial :} Par son maillage géographique et ses partenariats avec les collectivités locales, la banque contribue au développement économique des territoires, particulièrement dans les zones rurales et périurbaines.

\section{Organisation et structure institutionnelle}

\subsection{Gouvernance et direction générale}

La BNH est dirigée par un Directeur Général nommé par décret présidentiel, assisté d'un comité de direction composé des directeurs centraux. L'institution dispose d'un conseil d'administration présidé par le ministre chargé des Finances, garantissant l'alignement de sa stratégie sur les politiques publiques nationales.

\subsection{Organisation territoriale}

L'organisation territoriale de la BNH s'articule autour de :

\begin{itemize}
    \item \textbf{Le siège social} à Alger, qui centralise les fonctions stratégiques, financières et de pilotage
    \item \textbf{48 directions de wilayas} correspondant au découpage administratif national
    \item \textbf{Plus de 200 agences} réparties sur l'ensemble du territoire national
    \item \textbf{Des bureaux locaux} dans les communes rurales pour assurer la proximité avec les bénéficiaires
\end{itemize}

Cette organisation décentralisée permet à la banque d'assurer une présence de proximité tout en maintenant la cohérence de ses politiques d'attribution et de gestion des aides.

\subsection{Directions centrales}

La structure organisationnelle de la BNH comprend plusieurs directions centrales spécialisées :

\begin{itemize}
    \item Direction Générale
    \item Direction de l'Audit et du Contrôle
    \item Direction des Finances et de la Comptabilité
    \item Direction du Réseau et du Développement Commercial
    \item Direction des Ressources Humaines
    \item Direction Juridique et du Contentieux
    \item Direction des Systèmes d'Information
    \item Direction des Risques et de la Conformité
\end{itemize}

\section{Direction des Systèmes d'Information - Environnement de travail}

\subsection{Missions et responsabilités}

La Direction des Systèmes d'Information (DSI) de la BNH joue un rôle stratégique dans la modernisation et l'efficacité opérationnelle de l'institution. Ses missions principales incluent :

\medskip

\textbf{Pilotage de la transformation digitale :} Définition et mise en œuvre de la stratégie informatique de la banque, alignée sur ses objectifs métier et ses contraintes réglementaires.

\textbf{Gestion des infrastructures :} Administration et évolution du système d'information central, incluant les serveurs, les réseaux, et les systèmes de sauvegarde et de sécurité.

\textbf{Développement d'applications :} Conception, développement et maintenance des applications métier, interfaces utilisateur, et outils de reporting nécessaires aux activités de la banque.

\textbf{Support et formation :} Assistance technique aux utilisateurs finaux et formation sur les outils informatiques déployés au niveau du siège et du réseau d'agences.

\subsection{Architecture technique existante}

L'environnement technologique de la BNH s'appuie sur une architecture hybride combinant solutions propriétaires et outils open-source :

\medskip

\textbf{Système de gestion de base de données :} Oracle Database pour le stockage des données transactionnelles critiques, complété par des instances MySQL pour les applications secondaires.

\textbf{Serveurs d'applications :} Environnement Java EE déployé sur des serveurs Apache Tomcat et JBoss, permettant l'exécution des applications web métier.

\textbf{Outils de développement :} Stack de développement basée sur Java/J2EE, PHP pour certaines applications web, et utilisation de frameworks modernes (Spring, Hibernate).

\textbf{Infrastructure réseau :} Réseau privé virtuel (VPN) sécurisé reliant le siège aux agences, avec redondance et mécanismes de basculement automatique.

\textbf{Outils de reporting :} Utilisation de JasperReports pour la génération de rapports et de tableaux de bord, complétée par des solutions de Business Intelligence.

\section{Contexte des aides au logement rural}

\subsection{Politique nationale de l'habitat rural}

L'aide au logement rural constitue un pilier de la politique nationale d'aménagement du territoire et de développement rural en Algérie. Ces programmes visent à :

\begin{itemize}
    \item Améliorer les conditions de logement des populations rurales
    \item Encourager la fixation des populations dans les zones rurales
    \item Réduire l'exode rural vers les centres urbains
    \item Préserver l'équilibre démographique territorial
    \item Moderniser l'habitat traditionnel rural
\end{itemize}

\subsection{Types d'aides gérées par la BNH}

La BNH administre plusieurs dispositifs d'aide au logement rural, chacun ayant ses spécifications techniques et réglementaires :

\medskip

\textbf{Aide à la construction rurale :} Subvention forfaitaire pour la construction de logements neufs en zones rurales, avec des critères d'éligibilité basés sur les revenus et la composition familiale.

\textbf{Aide à l'amélioration de l'habitat rural :} Financement de travaux de réhabilitation et de mise aux normes des logements existants.

\textbf{Aide aux équipements annexes :} Subventions pour la réalisation d'équipements complémentaires (clôtures, puits, assainissement individuel).

\textbf{Prêts bonifiés :} Dispositifs de crédit à taux préférentiels pour compléter les subventions directes.

\subsection{Processus d'attribution et de suivi}

Le processus d'attribution des aides au logement rural suit un circuit administratif complexe impliquant plusieurs acteurs :

\begin{enumerate}
    \item Dépôt des dossiers par les bénéficiaires au niveau des agences BNH
    \item Instruction technique et vérification de l'éligibilité
    \item Validation par les commissions locales
    \item Programmation budgétaire et engagement des crédits
    \item Suivi de l'avancement des travaux
    \item Versement des tranches de financement
    \item Contrôle de conformité et réception des ouvrages
\end{enumerate}

Chaque étape génère des données qui alimentent les systèmes d'information et contribuent aux bilans de suivi des programmes.

\section{Enjeux de la digitalisation des processus bancaires}

\subsection{Contexte réglementaire}

La modernisation des processus bancaires en Algérie s'inscrit dans un contexte réglementaire en évolution, marqué par :

\begin{itemize}
    \item Les directives de la Banque d'Algérie sur la digitalisation des services financiers
    \item Les exigences de traçabilité et de reporting imposées par les autorités de tutelle
    \item La nécessité de conformité avec les standards internationaux de gouvernance bancaire
    \item L'obligation de transparence dans la gestion des fonds publics
\end{itemize}

\subsection{Défis technologiques et organisationnels}

La transformation digitale de la BNH fait face à plusieurs défis spécifiques :

\medskip

\textbf{Legacy systems :} Intégration des nouveaux outils avec les systèmes existants, souvent développés sur des technologies anciennes.

\textbf{Formation des utilisateurs :} Accompagnement du changement pour les équipes habituées aux processus manuels.

\textbf{Sécurité des données :} Mise en place de mesures de cybersécurité adaptées aux exigences du secteur bancaire.

\textbf{Continuité de service :} Assurance d'une disponibilité continue des services pendant les phases de migration.

\section{Conclusion}

La Banque Nationale de l'Habitat représente un acteur institutionnel majeur du secteur financier algérien, avec une mission sociale forte dans le domaine de l'habitat. La complexité de ses processus métier, particulièrement dans la gestion des aides au logement rural, justifie pleinement l'investissement dans des outils d'automatisation et de digitalisation.

\medskip

La Direction des Systèmes d'Information, dotée d'une infrastructure technique solide et d'équipes compétentes, offre un environnement propice au développement d'outils innovants. Le projet d'automatisation des bilans mensuels s'inscrit dans cette démarche de modernisation et d'amélioration de l'efficacité opérationnelle.

\medskip

Dans le chapitre suivant, nous analyserons en détail les processus existants de génération des bilans mensuels et identifierons les besoins spécifiques auxquels notre solution devra répondre.