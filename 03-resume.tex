\mychapter{0}{Résumé}

La Banque Nationale de l'Habitat (BNH), dans sa mission d'accompagnement des politiques publiques de logement, gère les programmes d'aides au logement rural nécessitant un suivi rigoureux et une production régulière de bilans mensuels détaillés. Le processus existant de génération de ces bilans présentait des limitations importantes : dépendance aux traitements manuels, risques d'erreurs élevés, délais de production incompatibles avec les exigences de réactivité moderne, et mobilisation excessive de ressources humaines qualifiées sur des tâches répétitives.

\medskip

Ce projet de fin d'études visait à concevoir et développer un outil d'automatisation de la génération des bilans mensuels relatifs aux aides au logement rural. La démarche adoptée a combiné une analyse approfondie des processus existants, une conception technique rigoureuse basée sur une architecture multicouches, et un développement suivant une méthodologie agile adaptée au contexte bancaire.

\medskip

La solution développée s'appuie sur une architecture Java EE avec Spring Boot, intégrant les systèmes Oracle existants via des connecteurs optimisés. L'outil comprend cinq modules principaux : extraction et intégration automatisées des données, moteur de calcul d'indicateurs configurable, génération de rapports avec templates personnalisables, interface web responsive, et système d'administration complet.

\medskip

Les résultats obtenus dépassent largement les objectifs initiaux : réduction de 95\% des délais de génération (de 15-20 jours à 2-3 heures), diminution de 93\% de la charge de travail nécessaire, amélioration de 90\% de la qualité des données, et taux de satisfaction utilisateur exceptionnel de 4,7/5. L'impact organisationnel s'étend au-delà des gains quantifiables, avec une transformation qualitative des pratiques de travail permettant aux équipes de se recentrer sur leurs missions d'analyse et de conseil à plus forte valeur ajoutée.

\vspace{1cm}

\noindent\rule[2pt]{\textwidth}{0.5pt}

{\textbf{Mots clés :}}
Automatisation bancaire, Génération de bilans, Aides au logement rural, Java EE, Spring Boot, Oracle, Digitalisation des processus, BNH.
\\
\noindent\rule[2pt]{\textwidth}{0.5pt}

\clearpage

\mychapter{0}{Abstract}

The National Housing Bank (BNH), in its mission to support public housing policies, manages rural housing aid programs requiring rigorous monitoring and regular production of detailed monthly reports. The existing process for generating these reports had significant limitations: dependence on manual processing, high error risks, production delays incompatible with modern responsiveness requirements, and excessive mobilization of qualified human resources on repetitive tasks.

\medskip

This final year project aimed to design and develop an automation tool for generating monthly reports related to rural housing assistance. The approach adopted combined an in-depth analysis of existing processes, rigorous technical design based on multi-layered architecture, and development following an agile methodology adapted to the banking context.

\medskip

The developed solution is based on a Java EE architecture with Spring Boot, integrating existing Oracle systems via optimized connectors. The tool comprises five main modules: automated data extraction and integration, configurable indicator calculation engine, report generation with customizable templates, responsive web interface, and complete administration system.

\medskip

The results obtained far exceed the initial objectives: 95\% reduction in generation delays (from 15-20 days to 2-3 hours), 93\% reduction in required workload, 90\% improvement in data quality, and exceptional user satisfaction rate of 4.7/5. The organizational impact extends beyond quantifiable gains, with a qualitative transformation of work practices enabling teams to refocus on their higher value-added analysis and consulting missions.

\vspace{1cm}

\noindent\rule[2pt]{\textwidth}{0.5pt}

{\textbf{Keywords :}}
Banking automation, Report generation, Rural housing aid, Java EE, Spring Boot, Oracle, Process digitalization, BNH.
\\
\noindent\rule[2pt]{\textwidth}{0.5pt}

\chapter*{\hfill \begin{Arabic} ملخص \end{Arabic}}

\begin{Arabic}
    \addcontentsline{toc}{chapter}{ ملخص}
\end{Arabic}

\begin{Arabic}
    تدير البنك الوطني للسكن، في إطار مهمته لدعم سياسات الإسكان العامة، برامج المساعدة على السكن الريفي التي تتطلب متابعة دقيقة وإنتاج منتظم لتقارير شهرية مفصلة. كانت العملية الحالية لإنشاء هذه التقارير تعاني من قيود مهمة: الاعتماد على المعالجة اليدوية، مخاطر أخطاء عالية، تأخير في الإنتاج غير متوافق مع متطلبات الاستجابة الحديثة، وتعبئة مفرطة للموارد البشرية المؤهلة في المهام المتكررة.
\end{Arabic}

\medskip

\begin{Arabic}
    هدف مشروع نهاية الدراسة هذا إلى تصميم وتطوير أداة أتمتة لإنشاء التقارير الشهرية المتعلقة بمساعدات السكن الريفي. جمعت المنهجية المعتمدة بين تحليل معمق للعمليات الحالية، وتصميم تقني صارم قائم على هندسة متعددة الطبقات، وتطوير يتبع منهجية رشيقة مكيفة مع السياق المصرفي.
\end{Arabic}

\medskip

\begin{Arabic}
    يعتمد الحل المطور على هندسة Java EE مع Spring Boot، مع دمج أنظمة Oracle الحالية عبر موصلات محسنة. تشمل الأداة خمس وحدات رئيسية: استخراج البيانات وتكاملها التلقائي، محرك حساب المؤشرات القابل للتكوين، إنشاء التقارير مع قوالب قابلة للتخصيص، واجهة ويب متجاوبة، ونظام إدارة كامل.
\end{Arabic}

\medskip

\begin{Arabic}
    تتجاوز النتائج المحققة الأهداف الأولية إلى حد كبير: تقليل 95٪ في تأخير التوليد (من 15-20 يوماً إلى 2-3 ساعات)، انخفاض 93٪ في عبء العمل المطلوب، تحسن 90٪ في جودة البيانات، ومعدل رضا المستخدمين الاستثنائي 4.7/5. يمتد التأثير التنظيمي إلى ما وراء المكاسب القابلة للقياس، مع تحول نوعي في ممارسات العمل يمكن الفرق من إعادة التركيز على مهام التحليل والاستشارة ذات القيمة المضافة العالية.
\end{Arabic}

\vspace{3cm}

\noindent\rule[2pt]{\textwidth}{0.5pt}

\begin{Arabic}
    \textbf{كلمات مفتاحية :}

    أتمتة مصرفية، توليد التقارير، مساعدات السكن الريفي، Java EE، Spring Boot، Oracle، رقمنة العمليات، البنك الوطني للسكن.

\end{Arabic}

\noindent\rule[2pt]{\textwidth}{0.5pt}