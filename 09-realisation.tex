\chapter{Développement et réalisation}
\clearpage
\label{chap:realisation}

\section{Introduction}

Ce chapitre détaille la phase de développement et de réalisation de l'outil d'automatisation des bilans mensuels pour la BNH. Nous y présentons la méthodologie de développement adoptée, les technologies mises en œuvre, les phases de développement, ainsi que les principales fonctionnalités réalisées. Une attention particulière est accordée aux aspects d'intégration avec l'écosystème existant et aux procédures de tests mises en place.

\section{Méthodologie de développement adoptée}

\subsection{Choix méthodologique}

Pour ce projet, nous avons adopté une approche inspirée de la méthodologie Agile, adaptée aux contraintes de l'environnement bancaire et aux exigences de sécurité de la BNH. Cette approche hybride combine la flexibilité des méthodes agiles avec la rigueur documentaire nécessaire dans le secteur financier.

\medskip

La méthodologie retenue présente les caractéristiques suivantes :

\textbf{Développement itératif :} Le projet est découpé en quatre itérations de trois semaines chacune, permettant une validation progressive des fonctionnalités par les utilisateurs métier.

\textbf{Implication utilisateur continue :} Organisation de sessions de démonstration hebdomadaires avec les utilisateurs finaux pour valider les développements et ajuster les spécifications.

\textbf{Documentation allégée mais structurée :} Production de la documentation technique essentielle sans formalisme excessif, privilégiant les artefacts vivants (code documenté, tests automatisés).

\textbf{Tests continus :} Intégration des tests unitaires et d'intégration dans le processus de développement pour garantir la qualité du livrable final.

\subsection{Organisation du travail}

L'organisation du travail s'articule autour de plusieurs rôles et responsabilités :

\medskip

\textbf{Product Owner :} M. Salim Zeghdoudi, responsable de la définition des priorités fonctionnelles et de la validation des livrables.

\textbf{Scrum Master :} Mme. Hakima Naneche, facilitatrice pour la résolution des blocages et l'interface avec les équipes techniques.

\textbf{Développeur :} Rôle principal assumé par le stagiaire, responsable de la conception technique et du développement.

\textbf{Testeurs métier :} Équipe de chargés d'études participant aux tests d'acceptation et à la validation fonctionnelle.

\subsection{Planning de développement}

Le planning de développement s'organise en quatre sprints principaux :

\begin{table}[h]
    \centering
    \begin{tabular}{|l|l|l|}
        \hline
        \textbf{Sprint} & \textbf{Durée} & \textbf{Objectifs principaux}                           \\ \hline
        Sprint 1        & 3 semaines     & Infrastructure, authentification, extraction de données \\ \hline
        Sprint 2        & 3 semaines     & Moteur de calcul, validation des données                \\ \hline
        Sprint 3        & 3 semaines     & Génération de rapports, interfaces utilisateur          \\ \hline
        Sprint 4        & 3 semaines     & Tests d'intégration, documentation, déploiement         \\ \hline
    \end{tabular}
    \caption{Planning des sprints de développement.}
    \label{tab:planning-sprints}
\end{table}

Chaque sprint se décompose en plusieurs phases :
\begin{itemize}
    \item Planification et affinage du backlog (1 jour)
    \item Développement et tests unitaires (12 jours)
    \item Tests d'intégration et validation (4 jours)
    \item Démonstration et rétrospective (1 jour)
\end{itemize}

\section{Technologies et outils utilisés}

\subsection{Environnement de développement}

L'environnement de développement mis en place intègre les outils suivants :

\medskip

\textbf{IDE de développement :} Eclipse IDE 2021-06 avec les plugins Spring Tools Suite pour faciliter le développement d'applications Spring.

\textbf{Serveur d'application local :} Apache Tomcat 9.0.50 pour les tests de développement et l'intégration continue.

\textbf{Base de données de développement :} Oracle Database 12c Express Edition configurée avec un schéma miroir de la production pour les tests.

\textbf{Gestionnaire de versions :} Git avec repository hébergé sur le serveur GitLab interne de la BNH.

\textbf{Outil de build :} Apache Maven 3.8.1 pour la gestion des dépendances et l'automatisation du processus de construction.

\subsection{Stack technologique détaillée}

La stack technologique implémentée se compose des éléments suivants :

\medskip

\textbf{Backend :}
\begin{itemize}
    \item Java 8 - Langage principal de développement
    \item Spring Boot 2.5.4 - Framework principal pour l'inversion de contrôle
    \item Spring MVC - Pour l'architecture Model-View-Controller
    \item Spring Security - Gestion de l'authentification et des autorisations
    \item Spring Data JPA - Abstraction pour l'accès aux données
    \item Hibernate 5.4.32 - ORM pour la persistance des données
\end{itemize}

\textbf{Frontend :}
\begin{itemize}
    \item HTML5/CSS3 - Structure et style des pages web
    \item JavaScript ES6+ - Logique côté client
    \item Bootstrap 4.6.0 - Framework CSS responsive
    \item jQuery 3.6.0 - Manipulation DOM et appels AJAX
    \item Chart.js 3.5.1 - Génération de graphiques interactifs
\end{itemize}

\textbf{Reporting :}
\begin{itemize}
    \item JasperReports 6.17.0 - Moteur de génération de rapports
    \item Apache PDFBox 2.0.24 - Manipulation des documents PDF
    \item Apache POI 5.0.0 - Génération de fichiers Excel
    \item JFreeChart 1.5.3 - Création de graphiques pour les rapports
\end{itemize}

\subsection{Configuration de l'environnement}

La configuration de l'environnement respecte les standards de l'infrastructure BNH :

\medskip

\textbf{Configuration Spring Boot :} Utilisation de profils Spring pour différencier les environnements de développement, test, et production.

\textbf{Configuration de base de données :} Connection pooling avec HikariCP pour optimiser les performances d'accès aux données Oracle.

\textbf{Configuration de sécurité :} Intégration avec l'Active Directory de la BNH via LDAP pour l'authentification centralisée.

\textbf{Configuration de logging :} Utilisation de Logback avec rotation automatique des fichiers de log et niveaux configurables par environnement.

\section{Phases de développement et planification}

\subsection{Sprint 1 : Infrastructure et extraction de données}

Le premier sprint se concentre sur la mise en place des fondations techniques du système.

\medskip

\textbf{Objectifs réalisés :}
\begin{itemize}
    \item Configuration de l'architecture Spring Boot multicouches
    \item Implémentation du système d'authentification LDAP
    \item Développement des DAO pour l'accès aux données Oracle existantes
    \item Création du module d'extraction de données avec gestion d'erreurs
    \item Mise en place des tests unitaires avec JUnit 5
\end{itemize}

\textbf{Livrables techniques :}
\begin{itemize}
    \item Classes de configuration Spring et profils d'environnement
    \item Module d'authentification intégré à Active Directory
    \item Services d'extraction de données depuis SGD et module comptable
    \item Couverture de tests unitaires > 80\% pour les couches développées
\end{itemize}

\textbf{Défis rencontrés :}
L'intégration avec les bases Oracle existantes a nécessité une analyse approfondie des schémas de données legacy pour identifier les relations entre les tables. La mise en place du mapping Hibernate a demandé plusieurs itérations pour optimiser les performances des requêtes.

\subsection{Sprint 2 : Moteur de calcul et validation}

Le deuxième sprint implémente la logique métier centrale du système.

\medskip

\textbf{Objectifs réalisés :}
\begin{itemize}
    \item Développement du moteur de calcul d'indicateurs configurable
    \item Implémentation des règles de validation métier
    \item Création du système de gestion d'erreurs et d'alertes
    \item Développement de l'API REST pour les services de calcul
    \item Optimisation des performances des requêtes complexes
\end{itemize}

\textbf{Architecture du moteur de calcul :}
Le moteur de calcul est conçu selon le pattern Strategy pour permettre l'ajout dynamique de nouveaux indicateurs. Chaque type de calcul est implémenté comme une stratégie spécialisée :

\begin{itemize}
    \item \texttt{IndicateurFinancierStrategy} : Calculs des montants et engagements
    \item \texttt{IndicateurStatistiqueStrategy} : Comptages et répartitions
    \item \texttt{IndicateurTemporelStrategy} : Comparaisons et évolutions
    \item \texttt{IndicateurGeographiqueStrategy} : Analyses spatiales
\end{itemize}

\textbf{Performances obtenues :}
Les optimisations mises en œuvre permettent de traiter un volume de 100,000 dossiers en moins de 2 minutes, respectant largement l'objectif de performance de 10 minutes pour un bilan complet.

\subsection{Sprint 3 : Génération de rapports et interfaces}

Le troisième sprint se concentre sur la génération de rapports et l'interface utilisateur.

\medskip

\textbf{Objectifs réalisés :}
\begin{itemize}
    \item Intégration de JasperReports avec templates personnalisables
    \item Développement de l'interface web responsive
    \item Implémentation des fonctionnalités d'export multi-formats
    \item Création du tableau de bord avec graphiques interactifs
    \item Développement du module d'historisation des bilans
\end{itemize}

\textbf{Templates de rapports :}
Cinq templates principaux ont été développés :
\begin{itemize}
    \item Bilan mensuel synthétique (2 pages)
    \item Bilan détaillé par type d'aide (5-8 pages)
    \item Rapport de répartition géographique avec cartes
    \item Tableau de bord des indicateurs clés
    \item Rapport d'analyse comparative multi-périodes
\end{itemize}

\textbf{Interface utilisateur :}
L'interface web comprend six modules principaux accessibles selon les profils utilisateur :
\begin{itemize}
    \item Module de génération de bilans
    \item Bibliothèque de documents
    \item Tableau de bord de monitoring
    \item Console d'administration
    \item Module de configuration
    \item Interface de consultation
\end{itemize}

\subsection{Sprint 4 : Tests et déploiement}

Le quatrième sprint finalise le projet avec les tests complets et le déploiement.

\medskip

\textbf{Objectifs réalisés :}
\begin{itemize}
    \item Tests d'intégration complets avec données de production anonymisées
    \item Tests de performance et de montée en charge
    \item Documentation technique et utilisateur
    \item Formation des utilisateurs finaux
    \item Déploiement en environnement de pré-production
\end{itemize}

\textbf{Stratégie de tests :}
La stratégie de tests adoptée couvre plusieurs niveaux :
\begin{itemize}
    \item Tests unitaires : 85\% de couverture de code
    \item Tests d'intégration : Validation des flux de données end-to-end
    \item Tests de performance : Validation des temps de réponse cibles
    \item Tests de sécurité : Validation des contrôles d'accès et de sécurisation
    \item Tests d'acceptation utilisateur : Validation fonctionnelle par les métiers
\end{itemize}

\section{Présentation de l'application développée}

\subsection{Architecture de l'application}

L'application finale s'appuie sur une architecture modulaire comprenant :

\medskip

\textbf{Module Core :} Services transverses (sécurité, configuration, logging)
\textbf{Module Data :} Gestion de l'accès aux données et intégration
\textbf{Module Business :} Logique métier et moteur de calcul
\textbf{Module Reporting :} Génération et export des bilans
\textbf{Module Web :} Interface utilisateur et API REST

\subsection{Fonctionnalités principales réalisées}

\textbf{Authentification et gestion des droits :}
\begin{itemize}
    \item Authentification via Active Directory de la BNH
    \item Gestion des rôles : Administrateur, Responsable, Chargé d'études, Consultant
    \item Contrôle d'accès granulaire par module et fonctionnalité
    \item Session timeout configurable et déconnexion automatique
\end{itemize}

\textbf{Extraction et intégration de données :}
\begin{itemize}
    \item Connecteurs vers SGD Oracle et module comptable
    \item Extraction incrémentale avec gestion des deltas
    \item Règles de validation et nettoyage automatique des données
    \item Gestion des erreurs avec retry automatique et notification
\end{itemize}

\textbf{Moteur de calcul d'indicateurs :}
\begin{itemize}
    \item 25 indicateurs prédéfinis couvrant les besoins identifiés
    \item Configuration flexible des formules de calcul
    \item Calculs en parallèle pour optimiser les performances
    \item Cache intelligent pour éviter les recalculs inutiles
\end{itemize}

\textbf{Génération de bilans :}
\begin{itemize}
    \item 5 modèles de bilans avec templates personnalisables
    \item Export en PDF, Excel, et CSV
    \item Génération de graphiques intégrés (barres, camemberts, courbes)
    \item Prévisualisation avant génération finale
\end{itemize}

\textbf{Interface web :}
\begin{itemize}
    \item Design responsive compatible mobile/tablette/desktop
    \item Tableau de bord avec indicateurs temps réel
    \item Assistant de génération guidée pas à pas
    \item Historique complet avec fonction de recherche avancée
\end{itemize}

\subsection{Captures d'écran des interfaces principales}

\begin{figure}[hbt!]
    \centering
    \includegraphics[width=14cm]{interface_tableau_bord.png}
    \caption{Interface principale - Tableau de bord.}
    \label{fig:interface-dashboard}
\end{figure}

Le tableau de bord principal présente une vue synthétique des informations clés avec :
\begin{itemize}
    \item Indicateurs de performance en temps réel
    \item Graphiques d'évolution des principaux KPI
    \item Accès rapides aux fonctions courantes
    \item Notifications et alertes système
\end{itemize}

\begin{figure}[hbt!]
    \centering
    \includegraphics[width=14cm]{interface_generation.png}
    \caption{Interface de génération de bilans.}
    \label{fig:interface-generation}
\end{figure}

L'interface de génération propose :
\begin{itemize}
    \item Assistant guidé en 4 étapes
    \item Paramétrage flexible des critères de sélection
    \item Prévisualisation des données sélectionnées
    \item Choix du template et options d'export
\end{itemize}

\section{Tests et validation}

\subsection{Tests unitaires et d'intégration}

Les tests développés couvrent l'ensemble des couches de l'application :

\medskip

\textbf{Tests unitaires :} 120 tests unitaires développés avec JUnit 5, couvrant 85\% du code métier. Les tests utilisent des mocks pour isoler les dépendances et garantir la fiabilité des tests.

\textbf{Tests d'intégration :} 35 tests d'intégration validant les flux complets, de l'extraction de données à la génération des bilans, avec une base de données de test contenantun échantillon représentatif de données de production.

\textbf{Tests de performance :} Validation des temps de réponse avec JMeter sur des volumes de données réalistes (50,000 à 100,000 dossiers).

\subsection{Tests d'acceptation utilisateur}

Les tests d'acceptation ont été menés avec trois groupes d'utilisateurs représentatifs :

\medskip

\textbf{Groupe 1 - Chargés d'études :} Validation des fonctionnalités de génération courante et de consultation des bilans.

\textbf{Groupe 2 - Responsables métier :} Validation des fonctions de paramétrage et de supervision.

\textbf{Groupe 3 - Direction :} Validation de l'ergonomie générale et de la pertinence des informations produites.

\textbf{Résultats des tests utilisateur :}
\begin{itemize}
    \item Taux de satisfaction global : 92\%
    \item Facilité d'utilisation : 88\%
    \item Pertinence des fonctionnalités : 95\%
    \item Performance perçue : 91\%
\end{itemize}

\section{Conclusion}

La phase de développement et réalisation a permis de concrétiser la conception théorique en une application opérationnelle répondant aux besoins identifiés. La méthodologie agile adoptée s'est révélée particulièrement adaptée au contexte du projet, permettant une validation continue des développements par les utilisateurs métier.

\medskip

L'application développée intègre toutes les fonctionnalités spécifiées dans le cahier des charges, avec des performances dépassant les objectifs initiaux. L'architecture modulaire choisie facilite la maintenance future et l'évolution du système selon les besoins émergents.

\medskip

Les tests complets réalisés, tant techniques qu'utilisateur, valident la robustesse de la solution et son adéquation aux processus métier de la BNH. La forte satisfaction exprimée par les utilisateurs témoigne de la réussite de l'approche centrée utilisateur adoptée tout au long du projet.

\medskip

Le chapitre suivant présente les résultats obtenus suite au déploiement de la solution, incluant les gains mesurés en termes d'efficacité opérationnelle et les retours d'expérience des utilisateurs en situation réelle d'utilisation.