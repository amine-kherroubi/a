\chapter{Glossaire et définitions}
\label{app:glossaire}

\section{Termes métier bancaire}

\subsection{Aide au logement rural}
Dispositif gouvernemental de soutien financier à la construction, réhabilitation ou amélioration de logements en zones rurales, géré par la BNH dans le cadre de sa mission de service public.

\subsection{AAPL (Aide à l'Accession à la Propriété du Logement)}
Programme de soutien gouvernemental facilitant l'accès à la propriété immobilière pour les ménages à revenus modestes et intermédiaires.

\subsection{BNH (Banque Nationale de l'Habitat)}
Établissement public à caractère économique et commercial, anciennement Caisse Nationale du Logement (CNL), spécialisé dans le financement de l'habitat en Algérie.

\subsection{CNL (Caisse Nationale du Logement)}
Ancien nom de la Banque Nationale de l'Habitat, avant sa transformation en 2018.

\subsection{Engagement budgétaire}
Acte par lequel un organisme public s'oblige financièrement envers un tiers dans la limite des crédits ouverts et des autorisations budgétaires.

\subsection{Mandatement}
Ordre de paiement émis par l'ordonnateur au comptable public pour effectuer le règlement d'une dépense préalablement engagée et liquidée.

\subsection{SGD (Système de Gestion des Dossiers)}
Application informatique centralisant la gestion des dossiers de demande d'aide au logement, développée sur technologie Oracle Forms.

\section{Termes techniques informatiques}

\subsection{API REST (Representational State Transfer)}
Architecture de services web utilisant les protocoles HTTP standard pour permettre la communication entre applications via des requêtes standardisées.

\subsection{DAO (Data Access Object)}
Pattern de conception permettant d'abstraire l'accès aux données en encapsulant les opérations CRUD dans des objets dédiés.

\subsection{ETL (Extract, Transform, Load)}
Processus d'intégration de données consistant à extraire des données de sources multiples, les transformer selon les règles métier, puis les charger dans un système cible.

\subsection{JasperReports}
Moteur open source de génération de rapports pour applications Java, permettant la création de documents PDF, Excel, HTML à partir de templates XML.

\subsection{JDBC (Java Database Connectivity)}
API Java standard permettant la connexion et l'exécution de requêtes sur des bases de données relationnelles.

\subsection{LDAP (Lightweight Directory Access Protocol)}
Protocole réseau pour accéder à un service d'annuaire distribué, utilisé notamment pour l'authentification centralisée des utilisateurs.

\subsection{Maven}
Outil de gestion de projet Java automatisant la compilation, les tests, la gestion des dépendances et le packaging des applications.

\subsection{ORM (Object-Relational Mapping)}
Technique de programmation permettant de faire correspondre les objets d'un langage orienté objet avec les tables d'une base de données relationnelle.

\subsection{Spring Framework}
Framework Java open source facilitant le développement d'applications enterprise via l'inversion de contrôle et la programmation orientée aspect.

\section{Indicateurs et métriques}

\subsection{KPI (Key Performance Indicator)}
Indicateur clé de performance permettant de mesurer l'efficacité d'une organisation ou d'un processus par rapport à ses objectifs.

\subsection{ROI (Return On Investment)}
Ratio financier mesurant la rentabilité d'un investissement en comparant les bénéfices obtenus au coût de l'investissement initial.

\subsection{SLA (Service Level Agreement)}
Accord contractuel définissant le niveau de service attendu entre un prestataire et son client, incluant des métriques de performance et de disponibilité.

\subsection{Taux d'adoption}
Pourcentage d'utilisateurs potentiels ayant effectivement adopté et utilisé régulièrement une solution technologique mise à leur disposition.

\section{Méthodologies et frameworks}

\subsection{Agile}
Ensemble de méthodes de développement logiciel privilégiant la collaboration, l'adaptation au changement et la livraison incrémentale de valeur.

\subsection{DevOps}
Culture et ensemble de pratiques visant à raccourcir le cycle de développement logiciel tout en assurant la qualité et la fiabilité des livraisons.

\subsection{MVC (Model-View-Controller)}
Pattern architectural séparant les données (Model), la présentation (View) et la logique de contrôle (Controller) d'une application.

\subsection{Scrum}
Framework agile pour la gestion de projet, organisant le travail en sprints avec des rôles définis (Product Owner, Scrum Master, équipe de développement).

\subsection{UML (Unified Modeling Language)}
Langage de modélisation graphique standardisé utilisé pour spécifier, visualiser et documenter les systèmes logiciels orientés objet.

\chapter{Documentation technique}
\label{app:documentation}

\section{Architecture système}

\subsection{Schéma d'architecture global}
\begin{figure}[hbt!]
    \centering
    \includegraphics[width=15cm]{architecture_detaillee.png}
    \caption{Architecture détaillée du système d'automatisation.}
    \label{fig:architecture-detaillee}
\end{figure}

L'architecture système s'organise autour de cinq couches principales :

\begin{itemize}
    \item \textbf{Couche Présentation} : Interfaces web HTML5/CSS3/JavaScript
    \item \textbf{Couche Application} : Services Spring Boot et contrôleurs REST
    \item \textbf{Couche Métier} : Logique applicative et moteur de calcul
    \item \textbf{Couche Persistance} : Accès aux données via Hibernate/JPA
    \item \textbf{Couche Données} : Bases Oracle existantes et nouveau schéma applicatif
\end{itemize}

\subsection{Matrice de traçabilité des exigences}

\begin{longtable}{|p{1cm}|p{6cm}|p{3cm}|p{3cm}|p{2cm}|}
    \hline
    \textbf{ID} & \textbf{Exigence}                      & \textbf{Composant}         & \textbf{Status} & \textbf{Tests} \\ \hline
    EX-001      & Extraction automatique des données SGD & DataExtractionService      & Implémenté      & OK             \\ \hline
    EX-002      & Calcul automatisé des indicateurs      & IndicatorCalculationEngine & Implémenté      & OK             \\ \hline
    EX-003      & Génération de bilans PDF               & ReportGeneratorService     & Implémenté      & OK             \\ \hline
    EX-004      & Interface web responsive               & Frontend HTML5/CSS3        & Implémenté      & OK             \\ \hline
    EX-005      & Authentification LDAP                  & SecurityModule             & Implémenté      & OK             \\ \hline
    EX-006      & Gestion des droits d'accès             & AuthorizationService       & Implémenté      & OK             \\ \hline
    EX-007      & Historisation des bilans               & BilansHistoryService       & Implémenté      & OK             \\ \hline
    EX-008      & Export multi-formats                   & ExportService              & Implémenté      & OK             \\ \hline
    EX-009      & Tableau de bord graphiques             & DashboardService           & Implémenté      & OK             \\ \hline
    EX-010      & Monitoring système                     & MonitoringModule           & Implémenté      & OK             \\ \hline
    \caption{Matrice de traçabilité des exigences fonctionnelles.}
    \label{tab:traçabilite-exigences}
\end{longtable}

\section{Configuration et déploiement}

\subsection{Configuration Spring Boot}

\textbf{Fichier application.yml (Production) :}
\begin{verbatim}
server:
  port: 8080
  servlet:
    context-path: /bnh-bilans
    
spring:
  datasource:
    url: jdbc:oracle:thin:@//oracle-prod:1521/BNHPROD
    username: ${DB_USERNAME}
    password: ${DB_PASSWORD}
    driver-class-name: oracle.jdbc.OracleDriver
    hikari:
      maximum-pool-size: 20
      minimum-idle: 5
      connection-timeout: 30000
      
  jpa:
    hibernate:
      ddl-auto: validate
    show-sql: false
    database-platform: org.hibernate.dialect.Oracle12cDialect
    
  security:
    ldap:
      urls: ldap://ad-server:389
      base: dc=bnh,dc=dz
      
logging:
  level:
    com.bnh.bilans: INFO
    org.springframework.security: DEBUG
  file:
    name: /var/log/bnh-bilans/application.log
    max-size: 10MB
    max-history: 30

app:
  reporting:
    output-directory: /data/bilans/output
    templates-directory: /data/bilans/templates
  extraction:
    batch-size: 1000
    max-retry-attempts: 3
\end{verbatim}

\subsection{Scripts de déploiement}

\textbf{Script de déploiement automatisé (deploy.sh) :}
\begin{verbatim}
#!/bin/bash

# Configuration
APP_NAME="bnh-bilans"
APP_VERSION="1.0.0"
TOMCAT_HOME="/opt/tomcat"
BACKUP_DIR="/backup/webapps"
LOG_FILE="/var/log/deploy.log"

echo "=== Début déploiement $APP_NAME v$APP_VERSION ===" >> $LOG_FILE

# Arrêt Tomcat
echo "Arrêt de Tomcat..." >> $LOG_FILE
sudo systemctl stop tomcat

# Sauvegarde ancienne version
if [ -d "$TOMCAT_HOME/webapps/$APP_NAME" ]; then
    echo "Sauvegarde ancienne version..." >> $LOG_FILE
    sudo mv "$TOMCAT_HOME/webapps/$APP_NAME" "$BACKUP_DIR/$APP_NAME-$(date +%Y%m%d_%H%M%S)"
fi

# Déploiement nouvelle version
echo "Déploiement nouvelle version..." >> $LOG_FILE
sudo cp target/$APP_NAME-$APP_VERSION.war $TOMCAT_HOME/webapps/$APP_NAME.war

# Configuration permissions
sudo chown tomcat:tomcat $TOMCAT_HOME/webapps/$APP_NAME.war

# Démarrage Tomcat
echo "Démarrage de Tomcat..." >> $LOG_FILE
sudo systemctl start tomcat

# Vérification déploiement
sleep 30
if curl -f -s "http://localhost:8080/$APP_NAME/actuator/health" > /dev/null; then
    echo "Déploiement réussi" >> $LOG_FILE
    exit 0
else
    echo "Échec du déploiement" >> $LOG_FILE
    exit 1
fi
\end{verbatim}

\section{Guide d'utilisation}

\subsection{Procédure de génération de bilan}

\textbf{Étape 1 : Connexion et authentification}
\begin{enumerate}
    \item Accéder à l'URL : http://bnh-server:8080/bnh-bilans
    \item Saisir identifiants Active Directory
    \item Vérifier les droits d'accès affichés
\end{enumerate}

\textbf{Étape 2 : Paramétrage du bilan}
\begin{enumerate}
    \item Cliquer sur "Génération de bilans"
    \item Sélectionner la période (mois/année)
    \item Choisir le type de bilan souhaité
    \item Définir les critères de filtrage (région, type d'aide)
    \item Valider les paramètres
\end{enumerate}

\textbf{Étape 3 : Lancement de la génération}
\begin{enumerate}
    \item Vérifier l'aperçu des données sélectionnées
    \item Choisir le format de sortie (PDF, Excel, CSV)
    \item Cliquer sur "Générer le bilan"
    \item Suivre la progression via la barre de statut
\end{enumerate}

\textbf{Étape 4 : Récupération du résultat}
\begin{enumerate}
    \item Notification automatique de fin de traitement
    \item Téléchargement du fichier généré
    \item Archivage automatique dans l'historique
    \item Possibilité de partage sécurisé
\end{enumerate}

\subsection{Procédure de maintenance}

\textbf{Maintenance quotidienne :}
\begin{itemize}
    \item Vérification logs applicatifs : /var/log/bnh-bilans/
    \item Contrôle espace disque serveur : df -h
    \item Vérification santé application : http://server:8080/bnh-bilans/actuator/health
    \item Sauvegarde base de données : script backup-db.sh
\end{itemize}

\textbf{Maintenance hebdomadaire :}
\begin{itemize}
    \item Rotation des logs avec logrotate
    \item Nettoyage fichiers temporaires
    \item Vérification performances base de données
    \item Test restauration sauvegarde
\end{itemize}

\textbf{Maintenance mensuelle :}
\begin{itemize}
    \item Mise à jour sécurité système
    \item Optimisation base de données
    \item Révision des configurations
    \item Test plan de reprise d'activité
\end{itemize}

\chapter{Captures d'écran détaillées}
\label{app:captures}

\section{Interface d'administration}

\begin{figure}[hbt!]
    \centering
    \includegraphics[width=14cm]{admin_users_management.png}
    \caption{Interface de gestion des utilisateurs.}
    \label{fig:admin-users}
\end{figure}

L'interface de gestion des utilisateurs permet :
\begin{itemize}
    \item Création/modification/suppression des comptes utilisateur
    \item Attribution des rôles et permissions
    \item Gestion des groupes d'accès
    \item Monitoring de l'activité utilisateur
    \item Export des logs d'audit
\end{itemize}

\begin{figure}[hbt!]
    \centering
    \includegraphics[width=14cm]{admin_system_config.png}
    \caption{Interface de configuration système.}
    \label{fig:admin-config}
\end{figure}

\section{Module de monitoring}

\begin{figure}[hbt!]
    \centering
    \includegraphics[width=14cm]{monitoring_dashboard.png}
    \caption{Tableau de bord de monitoring système.}
    \label{fig:monitoring}
\end{figure}

Le tableau de bord de monitoring affiche :
\begin{itemize}
    \item Métriques de performance en temps réel
    \item Utilisation des ressources système
    \item Statut des connexions base de données
    \item Historique des temps de réponse
    \item Alertes système actives
\end{itemize}

\section{Interface de consultation}

\begin{figure}[hbt!]
    \centering
    \includegraphics[width=14cm]{consultation_bilans.png}
    \caption{Interface de consultation des bilans historiques.}
    \label{fig:consultation}
\end{figure}

L'interface de consultation propose :
\begin{itemize}
    \item Recherche multicritères dans l'historique
    \item Prévisualisation intégrée des documents
    \item Comparaison entre plusieurs bilans
    \item Export et partage sécurisé
    \item Annotations collaboratives
\end{itemize}

\chapter{Métriques et statistiques}
\label{app:metriques}

\section{Indicateurs de performance technique}

\subsection{Temps de réponse par fonctionnalité}

\begin{longtable}{|l|c|c|c|c|}
    \hline
    \textbf{Fonctionnalité}    & \textbf{Temps moyen} & \textbf{Temps médian} & \textbf{95e percentile} & \textbf{Max observé} \\ \hline
    Authentification           & 0.8s                 & 0.7s                  & 1.2s                    & 2.1s                 \\ \hline
    Extraction données         & 45s                  & 42s                   & 78s                     & 120s                 \\ \hline
    Calcul indicateurs         & 25s                  & 22s                   & 48s                     & 65s                  \\ \hline
    Génération PDF             & 15s                  & 12s                   & 28s                     & 35s                  \\ \hline
    Export Excel               & 8s                   & 7s                    & 15s                     & 22s                  \\ \hline
    Chargement tableau de bord & 2.1s                 & 1.8s                  & 3.5s                    & 4.8s                 \\ \hline
    \caption{Métriques de performance par fonctionnalité.}
    \label{tab:performance-metriques}
\end{longtable}

\subsection{Utilisation des ressources système}

\textbf{Consommation mémoire :}
\begin{itemize}
    \item Mémoire moyenne : 2.1 GB
    \item Pic maximum observé : 3.8 GB
    \item Mémoire heap JVM : 70\% de 4GB alloués
    \item Garbage collection : < 2\% du temps total
\end{itemize}

\textbf{Utilisation CPU :}
\begin{itemize}
    \item Charge moyenne : 15\%
    \item Pic lors de génération : 85\%
    \item Temps de réponse maintenu < 5s même à 80\% CPU
    \item Scaling linéaire observé jusqu'à 20 utilisateurs simultanés
\end{itemize}

\section{Indicateurs d'usage}

\subsection{Statistiques d'utilisation mensuelle}

\begin{longtable}{|l|c|c|c|}
    \hline
    \textbf{Mois} & \textbf{Bilans générés} & \textbf{Utilisateurs actifs} & \textbf{Temps économisé} \\ \hline
    Janvier 2024  & 45                      & 12                           & 98 heures                \\ \hline
    Février 2024  & 52                      & 15                           & 112 heures               \\ \hline
    Mars 2024     & 48                      & 14                           & 105 heures               \\ \hline
    Avril 2024    & 61                      & 18                           & 128 heures               \\ \hline
    Mai 2024      & 58                      & 16                           & 122 heures               \\ \hline
    Juin 2024     & 55                      & 17                           & 118 heures               \\ \hline
    \caption{Évolution de l'usage mensuel du système.}
    \label{tab:usage-mensuel}
\end{longtable}

\subsection{Répartition par type de bilan}

\begin{itemize}
    \item Bilans mensuels synthétiques : 45\%
    \item Bilans détaillés par aide : 25\%
    \item Rapports géographiques : 15\%
    \item Analyses comparatives : 10\%
    \item Tableaux de bord : 5\%
\end{itemize}

\section{Indicateurs qualité}

\subsection{Taux d'erreur et incidents}

\textbf{Période : 6 premiers mois d'exploitation}
\begin{itemize}
    \item Incidents majeurs : 2 (indisponibilité > 1h)
    \item Incidents mineurs : 8 (dégradation performance)
    \item Taux d'erreur moyen : 1.2\%
    \item MTTR (temps moyen de résolution) : 45 minutes
    \item Disponibilité effective : 99.7\%
\end{itemize}

\subsection{Satisfaction utilisateur détaillée}

\textbf{Résultats enquête trimestrielle (25 répondants) :}
\begin{itemize}
    \item Facilité d'apprentissage : 4.3/5
    \item Efficacité d'utilisation : 4.8/5
    \item Mémorisation des procédures : 4.5/5
    \item Fréquence d'erreurs : 4.6/5 (peu d'erreurs)
    \item Satisfaction subjective : 4.7/5
    \item Recommandation à d'autres services : 94\%
\end{itemize}

\chapter{Références et bibliographie}
\label{app:references}

\section{Documentation technique consultée}

\begin{itemize}
    \item Oracle Corporation. (2021). \textit{Oracle Database 12c Administrator's Guide}. Oracle Press.
    \item Walls, C. (2020). \textit{Spring Boot in Action, Second Edition}. Manning Publications.
    \item Johnson, R. et al. (2019). \textit{Professional Java Development with the Spring Framework}. Wrox Press.
    \item Apache Software Foundation. (2021). \textit{Apache Tomcat 9 Configuration Reference}.
    \item Hibernate Team. (2021). \textit{Hibernate ORM User Guide}. Red Hat Inc.
\end{itemize}

\section{Standards et normes appliqués}

\begin{itemize}
    \item ISO/IEC 25010:2011 - Systems and software Quality Requirements and Evaluation (SQuaRE)
    \item ISO/IEC 27001:2013 - Information Security Management Systems
    \item WCAG 2.1 - Web Content Accessibility Guidelines
    \item IEEE 830-1998 - Recommended Practice for Software Requirements Specifications
    \item PMI PMBOK Guide - Project Management Body of Knowledge
\end{itemize}

\section{Réglementation bancaire de référence}

\begin{itemize}
    \item Règlement Banque d'Algérie n° 11-08 relatif au contrôle interne des banques et établissements financiers
    \item Instruction Banque d'Algérie n° 74-94 relative à la fixation des règles prudentielles
    \item Circulaire Banque d'Algérie n° 01-2021 relative à la sécurité des systèmes d'information bancaires
    \item Loi n° 18-05 relative à la protection des données à caractère personnel
\end{itemize}

\section{Articles et publications de référence}

\begin{itemize}
    \item Davenport, T. H. (2019). "The Future of Work in Technology-Mediated Financial Services". \textit{Harvard Business Review}, 97(3), 45-52.
    \item Chen, L. et al. (2020). "Digital Transformation in Banking: Challenges and Opportunities in Emerging Markets". \textit{Journal of Financial Technology}, 15(2), 23-38.
    \item Benlahcene, A. (2021). "Modernisation du secteur bancaire algérien : enjeux et perspectives". \textit{Revue d'Économie du Maghreb}, 28(4), 112-128.
    \item KPMG. (2022). \textit{Banking Automation: Global Survey Report}. KPMG International.
    \item McKinsey & Company. (2021). \textit{The Future of Bank Risk Management}. McKinsey Global Institute.
\end{itemize}